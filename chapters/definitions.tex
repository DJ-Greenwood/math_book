\chapter{Mathematical Definitions, Notations, and Concepts}

\section{Addition (+)}

\textbf{Definition:} The process of combining two or more numbers to find their total or sum.

\textbf{Notation:} \( a + b = c \), where \( a \) and \( b \) are addends, and \( c \) is the sum.

\section{Subtraction \textendash}

\textbf{Definition:} The process of taking one number away from another.

\textbf{Notation:} \( a - b = c \), where \( a \) is the minuend, \( b \) is the subtrahend, and \( c \) is the difference.

\section{Multiplication (\( \times \) or \( \cdot \))}

\textbf{Definition:} The process of combining equal groups. It is repeated addition.

\textbf{Notation:} \( a \times b = c \), where \( a \) and \( b \) are factors, and \( c \) is the product.

\section{Division (÷ or /)}

\textbf{Definition:} The process of splitting a number into equal parts or groups.

\textbf{Notation:} \( a \div b = c \), where \( a \) is the dividend, \( b \) is the divisor, and \( c \) is the quotient.

\section{Fractions}

\textbf{Definition:} Represents a part of a whole.

\textbf{Notation:} \( \frac{a}{b} \), where \( a \) is the numerator, and \( b \) is the denominator.

\section{Decimals}

\textbf{Definition:} Another way to represent fractions, especially those with denominators of 10.

\textbf{Notation:} \( 0.a \), where \( a \) represents digits after the decimal point.

\section{Percentage (\%)}

\textbf{Definition:} A fraction expressed as parts per hundred.

\textbf{Notation:} \( a\% \), where \( a \) is a part out of 100.

\section{Exponents \(\wedge\) or superscript}

\textbf{Definition:} A number that tells how many times the base number is multiplied by itself.

\textbf{Notation:} \( a^b \), where \( a \) is the base and \( b \) is the exponent.

\section{Roots (\(\sqrt{}\))}

\textbf{Definition:} A number that, when multiplied by itself a certain number of times, gives the original number.

\textbf{Notation:} \( \sqrt[b]{a} \) or \( a^{1/b} \), where \( a \) is the radicand and \( b \) is the degree of the root.

\section{Less than, greater than and inequalities ($<$, $>$, $\leq$, $\geq$)}

\textbf{Definition:} Mathematical statements that compare two values.

\textbf{Notation:} \( a < b \) (less than), \( a > b \) (greater than), \( a \leq b \) (less than or equal to), \( a \geq b \) (greater than or equal to).

\section{Order of Operations (PEMDAS)}

\textbf{Parentheses (P):} First, perform operations inside parentheses or brackets.

\textbf{Example:} \( (3 + 5) \times 2 = 16 \).

\textbf{Exponents (E):} Next, solve exponents or powers.

\textbf{Example:} \( 2^3 = 8 \).

\textbf{Multiplication and Division (MD):} Perform these operations from left to right.

\textbf{Example:} \( 8 \div 2 \times 4 = 16 \).

\textbf{Addition and Subtraction (AS):} Lastly, perform addition and subtraction from left to right.

\textbf{Example:} \( 10 - 4 + 2 = 8 \).

The acronym PEMDAS helps remember the order: Parentheses, Exponents, Multiplication, Division, Addition, Subtraction.
