
\chapter{Advanced Algebra – Functions, Graphs, and Equations}

\section{Introduction: The Power of Algebra in Solving Real-World Problems}
Now that you’ve built a strong foundation in basic algebra, we’re ready to explore advanced algebra concepts, such as functions, equations, and graphs. Algebra is about identifying and understanding relationships between numbers, but when those relationships get more complex, we need new tools to make sense of them. That’s where functions and graphs come in.

In this chapter, you’ll learn how to work with different types of equations, understand the concept of functions, and graph relationships between variables. These skills will help you solve more complex problems and visualize how numbers relate to each other.

\section{What Is a Function?}
A function is a rule that assigns exactly one output (result) for each input. You can think of it like a machine: You put in an input (like a number), the function processes it, and you get an output.

Functions are written like this:
\[ f(x) = 2x + 3 \]

In this function:
\begin{itemize}
    \item \( x \) is the input (the number you put in).
    \item \( f(x) \) is the output (what you get after applying the function rule).
    \item The rule is to multiply \( x \) by 2 and then add 3.
\end{itemize}

Example: If \( x = 4 \), then:
\[ f(4) = 2(4) + 3 = 8 + 3 = 11 \]
So, the output is 11 when the input is 4.

Another Example: If \( x = -2 \), then:
\[ f(-2) = 2(-2) + 3 = -4 + 3 = -1 \]

\section{Linear Functions}
A linear function is a type of function where the graph forms a straight line. Linear functions follow the form:
\[ f(x) = mx + b \]



Where:
\begin{itemize}
    \item \( m \) is the slope of the line, representing how steep it is.
    \item \( b \) is the y-intercept, the point where the line crosses the y-axis (when \( x = 0 \)).
\end{itemize}

Example of a Linear Function:
\[ f(x) = 2x + 1 \]
\begin{itemize}
    \item The slope is 2 (for every 1 unit increase in \( x \), \( f(x) \) increases by 2).
    \item The y-intercept is 1 (the line crosses the y-axis at the point (0,1)).
\end{itemize}

Graphing a Linear Function:
\begin{enumerate}
    \item Start by finding the y-intercept (\( b \)).
    \item Use the slope (\( m \)) to plot additional points.
    \item Draw a line through the points.
\end{enumerate}

Example: For \( f(x) = 2x + 1 \):
\begin{itemize}
    \item The y-intercept is 1, so start by plotting the point (0,1).
    \item The slope is 2, meaning for every 1 unit you move to the right, move 2 units up. So plot the point (1,3).
    \item Connect the points with a straight line.
\end{itemize}

\section{Solving Linear Equations}
A linear equation is an equation that makes a straight line when graphed. Solving a linear equation means finding the value of the variable (usually \( x \)) that makes the equation true.

Example: Solve \( 3x + 4 = 13 \).
\begin{enumerate}
    \item Subtract 4 from both sides:
    \[ 3x = 9 \]
    \item Divide both sides by 3:
    \[ x = 3 \]
\end{enumerate}
The solution is \( x = 3 \).

\section{Quadratic Functions}
A quadratic function is more complex than a linear function and has the form:
\[ f(x) = ax^2 + bx + c \]

Where \( a \), \( b \), and \( c \) are constants.

The graph of a quadratic function is a curve called a parabola. Parabolas can open upward or downward, depending on the sign of \( a \):
\begin{itemize}
    \item If \( a > 0 \), the parabola opens upward.
    \item If \( a < 0 \), the parabola opens downward.
\end{itemize}

Example of a Quadratic Function:
\[ f(x) = x^2 - 4x + 3 \]

Graphing a Quadratic Function:
\begin{enumerate}
    \item Find the vertex, which is the highest or lowest point of the parabola.
    \item Plot the vertex and other points on both sides to create a symmetric curve.
\end{enumerate}

\section{Factoring Quadratic Equations}
One method for solving quadratic equations is factoring. This involves rewriting the quadratic equation in a way that allows you to find the values of \( x \) that make the equation true.

Example: Solve \( x^2 - 5x + 6 = 0 \) by factoring.
\begin{enumerate}
    \item Factor the quadratic expression:
    \[ x^2 - 5x + 6 = (x - 2)(x - 3) \]
    \item Set each factor equal to 0:
    \[ x - 2 = 0 \quad \text{or} \quad x - 3 = 0 \]
    \item Solve for \( x \):
    \[ x = 2 \quad \text{or} \quad x = 3 \]
\end{enumerate}
So, the solutions are \( x = 2 \) and \( x = 3 \).

\section{Graphing Quadratic Functions}
Graphing a quadratic function involves finding its key features:
\begin{enumerate}
    \item Vertex: The highest or lowest point of the parabola.
    \item Axis of symmetry: A vertical line that runs through the vertex and divides the parabola into two symmetrical halves.
    \item Intercepts: Points where the graph crosses the x-axis and y-axis.
\end{enumerate}

Example: Graph the quadratic function \( f(x) = x^2 - 4x + 3 \).
\begin{enumerate}
    \item Find the vertex.
    \begin{itemize}
        \item The vertex is at (2, -1).
    \end{itemize}
    \item Find the y-intercept.
    \begin{itemize}
        \item Set \( x = 0 \): \( f(0) = 0^2 - 4(0) + 3 = 3 \).
    \end{itemize}
    \item Plot the points and draw the parabola.
\end{enumerate}

\section{Solving Systems of Linear Equations}
A system of equations is a set of two or more equations that you solve together. A common method is to solve for one variable and substitute it into the other equation.

Example: Solve the system:
\[
\begin{cases}
2x + y = 5 \\
x - y = 1
\end{cases}
\]
\begin{enumerate}
    \item Solve the second equation for \( x \):
    \[ x = y + 1 \]
    \item Substitute \( x = y + 1 \) into the first equation:
    \[ 2(y + 1) + y = 5 \]
    \item Solve for \( y \):
    \[ 2y + 2 + y = 5 \quad \Rightarrow \quad 3y + 2 = 5 \quad \Rightarrow \quad 3y = 3 \quad \Rightarrow \quad y = 1 \]
    \item Substitute \( y = 1 \) back into \( x = y + 1 \):
    \[ x = 1 + 1 = 2 \]
\end{enumerate}
So, the solution is \( x = 2 \) and \( y = 1 \).

\section{Practice Makes Perfect: Let’s Try Some Exercises!}
\subsection*{Functions}
\begin{enumerate}
    \item Evaluate \( f(x) = 3x + 2 \) when \( x = 5 \).
    \item If \( f(x) = 2x^2 - 3x + 1 \), what is \( f(2) \)?
\end{enumerate}

\subsection*{Solving Equations}
\begin{enumerate}
    \item Solve \( 4x - 7 = 9 \).
    \item Solve \( x^2 - 6x + 8 = 0 \) by factoring.
\end{enumerate}

\subsection*{Graphing}
\begin{enumerate}
    \item Graph the linear function \( f(x) = -x + 2 \).
    \item Graph the quadratic function \( f(x) = x^2 + 2x - 3 \).
\end{enumerate}

\section{Real-Life Applications of Functions and Graphs}
\begin{itemize}
    \item \textbf{Business and Economics:} Companies use linear and quadratic equations to model profit and cost functions. By graphing these functions, they can find the optimal production levels to maximize profit.
    \item \textbf{Physics:} Linear and quadratic functions describe the motion of objects, such as the trajectory of a ball when you throw it. Understanding how to graph these functions helps predict where the ball will land.
    \item \textbf{Engineering:} Engineers use functions to model relationships between variables, such as the stress and strain on materials. Graphing these functions helps them design safer structures.
\end{itemize}

\section{Chapter Summary}
\begin{itemize}
    \item A function is a rule that assigns one output for each input. Linear functions form straight lines, while quadratic functions form parabolas.
    \item The equation of a linear function is \( f(x) = mx + b \), where \( m \) is the slope and \( b \) is the y-intercept.
    \item Quadratic functions take the form \( f(x) = ax^2 + bx + c \) and can be graphed as parabolas.
    \item You can solve linear equations by isolating the variable, and solve quadratic equations by factoring or graphing.
    \item Systems of equations involve solving two or more equations together to find where their graphs intersect.
\end{itemize}

\section*{Challenge Question}
Graph the function \( f(x) = 2x^2 - 4x + 1 \), find the vertex, and identify where the graph crosses the x-axis and y-axis.