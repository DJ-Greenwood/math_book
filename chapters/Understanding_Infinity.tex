
\chapter{Understanding Infinity – Limits and Advanced Topics}

\section{Introduction: Exploring the Concept of Infinity}
The idea of infinity is one of the most profound and mind-expanding concepts in mathematics. Unlike a regular number, infinity represents something that never ends or something that is larger than any number you can imagine. While infinity seems abstract, it is a central idea in many advanced math topics, from calculus to set theory and beyond.

In this chapter, we will explore the concept of limits, which help us understand how numbers behave as they approach infinity (or other values). We will also explore other fascinating advanced topics related to infinity, such as infinite sequences, series, and the nature of different "sizes" of infinity.

\section{What Is a Limit?}
In mathematics, a limit describes the value that a function or sequence "approaches" as the input (or index) gets closer to a certain point or as it heads toward infinity.

\subsection{Understanding Limits}
A limit helps us answer questions like:
\begin{itemize}
    \item "What happens to a function as $x$ gets larger and larger?"
    \item "What value does a function approach as $x$ gets closer to a particular number?"
\end{itemize}

\textbf{Example:} Consider the function $f(x) = \frac{1}{x}$. As $x$ gets larger and larger, the value of $f(x)$ gets closer to 0. We say:
\[
\lim_{x \to \infty} \frac{1}{x} = 0
\]
This means that as $x$ approaches infinity, the value of $f(x)$ approaches 0, but it never actually reaches 0.

\subsection{Left-Hand and Right-Hand Limits}
Sometimes, we’re interested in how a function behaves as it approaches a specific value from the left or the right.
\begin{itemize}
    \item The left-hand limit looks at how the function behaves as the input approaches from values smaller than the target value.
    \item The right-hand limit looks at how the function behaves as the input approaches from values larger than the target value.
\end{itemize}

\textbf{Example:} For the function $f(x) = \frac{1}{x-1}$, we might ask what happens as $x$ approaches 1 from both sides:
\[
\lim_{x \to 1^-} \frac{1}{x-1} = -\infty \quad \text{and} \quad \lim_{x \to 1^+} \frac{1}{x-1} = \infty
\]
This means that as $x$ approaches 1 from the left, $f(x)$ goes to negative infinity, and as $x$ approaches from the right, $f(x)$ goes to positive infinity.

\section{Understanding Infinite Sequences}
An infinite sequence is an ordered list of numbers that continues forever. Each number in the sequence is called a term, and the position of the term in the sequence is called its index.

\textbf{Example of an Infinite Sequence:}
Consider the sequence:
\[
1, \frac{1}{2}, \frac{1}{3}, \frac{1}{4}, \dots
\]
This sequence continues indefinitely, and the general term for the sequence is $\frac{1}{n}$, where $n$ is the index of the term.

As $n$ gets larger, the terms in the sequence get smaller, and we can use limits to describe this behavior:
\[
\lim_{n \to \infty} \frac{1}{n} = 0
\]
This means that as $n$ approaches infinity, the terms in the sequence approach 0.

\subsection{Converging vs. Diverging Sequences}
\begin{itemize}
    \item A sequence converges if the terms get closer and closer to a particular value.
    \item A sequence diverges if the terms do not approach a specific value and instead grow without bound or oscillate.
\end{itemize}

\textbf{Example:}
\begin{itemize}
    \item The sequence $\frac{1}{n}$ converges to 0 as $n \to \infty$.
    \item The sequence $n^2$ diverges because the terms grow larger and larger without bound as $n \to \infty$.
\end{itemize}

\section{What Is an Infinite Series?}
An infinite series is the sum of the terms of an infinite sequence. Just as with sequences, some series converge (add up to a finite value), while others diverge (grow without bound).

\textbf{Example of an Infinite Series:}
Consider the series:
\[
S = 1 + \frac{1}{2} + \frac{1}{4} + \frac{1}{8} + \dots
\]
This is called a geometric series because each term is a fixed fraction of the previous term. In this case, each term is half of the previous one.

We can find the sum of this geometric series by using the formula:
\[
S = \frac{a}{1 - r}
\]
Where:
\begin{itemize}
    \item $a$ is the first term,
    \item $r$ is the common ratio between terms (in this case, $r = \frac{1}{2}$).
\end{itemize}

For the series above:
\[
S = \frac{1}{1 - \frac{1}{2}} = \frac{1}{\frac{1}{2}} = 2
\]
So, the sum of this infinite series is 2, meaning the series converges.

\subsection{Diverging Series}
Not all infinite series converge. For example, the series:
\[
1 + 1 + 1 + 1 + \dots
\]
clearly grows without bound, so it diverges.

\section{Understanding Asymptotes}
An asymptote is a line that a graph approaches but never touches. Asymptotes often appear in graphs of functions that involve infinity, helping us understand the behavior of a function as it gets closer to certain points or heads toward infinity.

\subsection{Vertical Asymptote}
A vertical asymptote occurs when a function increases or decreases without bound as it approaches a specific $x$-value.

\textbf{Example:} The function $f(x) = \frac{1}{x-2}$ has a vertical asymptote at $x = 2$, because the function approaches infinity as $x$ gets closer to 2 from the right, and negative infinity as $x$ approaches from the left.

\subsection{Horizontal Asymptote}
A horizontal asymptote describes the behavior of a function as $x$ heads toward infinity. It represents a value that the function gets closer to but never actually reaches.

\textbf{Example:} For $f(x) = \frac{1}{x}$, the horizontal asymptote is $y = 0$, because as $x \to \infty$, the function approaches 0.

\section{Different Sizes of Infinity}
One of the most surprising facts about infinity is that not all infinities are the same size! This concept was first explored by the mathematician Georg Cantor, who showed that some infinite sets are "larger" than others.

\subsection{Countable vs. Uncountable Infinity}
\begin{itemize}
    \item A set is countably infinite if its elements can be put in one-to-one correspondence with the natural numbers. For example, the set of all whole numbers $\{0, 1, 2, 3, \dots\}$ is countably infinite.
    \item A set is uncountably infinite if it cannot be listed in this way. For example, the set of all real numbers between 0 and 1 is uncountably infinite, meaning that this set is "larger" than the set of natural numbers.
\end{itemize}

Cantor’s work showed that even within infinity, there are different levels of "size," revolutionizing our understanding of math.

\section{Practice Makes Perfect: Let’s Try Some Exercises!}
\subsection{Limits}
\begin{enumerate}
    \item Find $\lim_{x \to \infty} \frac{1}{x+3}$.
    \item Evaluate $\lim_{x \to 2} \frac{1}{x-2}$.
\end{enumerate}

\subsection{Sequences}
\begin{enumerate}
    \item Determine whether the sequence $\frac{1}{n^2}$ converges or diverges as $n \to \infty$.
    \item Find the limit of the sequence $2^n$ as $n \to \infty$.
\end{enumerate}

\subsection{Series}
\begin{enumerate}
    \item Determine if the series $1 + \frac{1}{2} + \frac{1}{3} + \dots$ converges or diverges.
    \item Find the sum of the geometric series $2 + 1 + \frac{1}{2} + \frac{1}{4} + \dots$.
\end{enumerate}

\section{Real-Life Applications of Infinity and Limits}
\begin{itemize}
    \item \textbf{Physics:} Limits and infinity are crucial in physics, especially in understanding concepts like velocity, acceleration, and forces that approach infinite values in certain scenarios (such as black holes).
    \item \textbf{Engineering:} Engineers use limits to analyze systems where quantities get very large or very small, such as the flow of fluids or the behavior of electrical currents in circuits.
    \item \textbf{Economics:} In economics, limits and series are used to model long-term growth, interest rates, and investment returns over infinite time horizons.
\end{itemize}

\section{Chapter Summary}
\begin{itemize}
    \item Limits help us understand how functions behave as they approach specific points or infinity.
    \item Infinite sequences are lists of numbers that go on forever, and they can either converge to a specific value or diverge.
    \item Infinite series are sums of the terms of infinite sequences, and some series converge to a finite value while others diverge.
    \item Asymptotes describe how a function behaves as it approaches certain values or infinity.
    \item There are different "sizes" of infinity, with countable and uncountable infinities representing different levels of infinity.
\end{itemize}

\section*{Challenge Question}
Prove that the harmonic series $1 + \frac{1}{2} + \frac{1}{3} + \frac{1}{4} + \dots$ diverges, meaning it grows without bound as more terms are added.