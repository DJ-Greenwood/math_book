\chapter{Patterns and Relationships – Introduction to Algebra}
\section{Introduction: The Power of Variables and Equations}
In the previous chapters, we explored numbers, addition, subtraction, multiplication, devision, fractions, decimals, and percentages. Now, we are ready to enter the world of algebra. Algebra may seem intimidating at first, but it's simply a way of using letters and symbols to represent numbers and relationships between them.

Why is algebra important? Because it allows us to solve problems where we don't know all the information yet. Whether you're calculating the price of multiple items, planning how to save money, or figuring out how far you've traveled, algebra helps you break these problems down and find solutions.

\section{What Is a Variable?}
A variable is a symbol (often a letter) that represents a number we don’t know yet. Think of it as a blank space that will eventually be filled in.

For example:

If you don’t know how many apples you have, you can use $x$ to represent the number of apples.
Here’s how it looks:

\begin{quote}
"I have $x$ apples."
\end{quote}

Using Variables in Equations: You can use variables to create equations, which are like math sentences that describe relationships between numbers.

For example:

You buy 3 apples, and you already had $x$ apples. The equation would look like this:
\[ x + 3 \]
(This means "the number of apples I already had plus 3 more apples.")

\section{What Is an Equation?}
An equation is a statement that two things are equal. It has two sides, usually with an equal sign (=) in the middle.

Here’s an example:

\[ x + 3 = 7 \]
This equation says that if you add 3 to some unknown number $x$, you will get 7.

Your job using algebra is often to solve the equation—to figure out what number $x$ represents.

\section{Solving Simple Equations}
The key to solving equations is to isolate the variable on one side of the equation. Let’s go back to the equation $x + 3 = 7$.

\begin{enumerate}
    \item Subtract 3 from both sides.
    \begin{quote}
    Why? Because you want to get $x$ by itself.
    \[ x + 3 - 3 = 7 - 3 \]
    \end{quote}
    \item Simplify.
    \[ x = 4 \]
\end{enumerate}

Now, we know that $x$ equals 4. This means you originally had 4 apples before buying 3 more.

\section{Understanding Patterns and Relationships}
Algebra is also about recognizing patterns and relationships between numbers. For example, if you know that the number of apples increases by 3 every day, you could describe this pattern with an equation.

Let’s say:

\begin{itemize}
    \item $x$ represents the number of apples you have.
    \item $d$ represents the number of days.
\end{itemize}

Each day, you get 3 more apples, so you can write:

\[ x = 3d \]
This equation shows that the number of apples ($x$) is equal to 3 times the number of days ($d$).
\begin{quote}
    Let $d = 5$ (the number of days).
    Then, using the equation $x = 3d$:
    \[ x = 3 \times 5 \]
    \[ x = 15 \]
\end{quote}

\section{Working with Word Problems in Algebra}
Let’s take a word problem and see how algebra helps us solve it:

\textbf{Problem:} Sarah has 5 books, and she buys 2 more books each week. How many books will Sarah have after 4 weeks?

\begin{enumerate}
    \item Identify what you know.
    \begin{itemize}
        \item Sarah starts with 5 books.
        \item She buys 2 more books every week.
        \item The number of weeks is 4.
    \end{itemize}
    \item Write the equation.
    \begin{quote}
    Let $x$ represent the number of books after 4 weeks.
    The total number of books is the 5 she started with, plus 2 books for each week:
    \[ x = 5 + 2 \times 4 \]
    \end{quote}
    \item Solve the equation.
    \[ x = 5 + 8 \]
    \[ x = 13 \]
\end{enumerate}

So, after 4 weeks, Sarah will have 13 books.

\section{The Distributive Property in Algebra}
In algebra, you’ll often come across expressions that use parentheses. When this happens, you can use the distributive property to simplify the expression.

The distributive property says:

\[ a(b + c) = ab + ac \]

For example:

\[ 2(3 + 4) \]
Using the distributive property, we can multiply 2 by both 3 and 4:
\[ 2(3 + 4) = 2 \times 3 + 2 \times 4 \]
\[ = 6 + 8 \]
\[ = 14 \]
This is helpful when solving more complex equations.

\section{Combining Like Terms}
Sometimes in algebra, you'll have similar terms that can be combined. These are called like terms. Like terms have the same variable raised to the same power.

Example:

\[ 3x + 4x \]
Both terms have the variable $x$, so you can add them together:
\[ 3x + 4x = 7x \]
This makes the equation simpler and easier to work with.

\section{Practice Makes Perfect: Let’s Try Some Exercises!}
\textbf{Solving Simple Equations:}
\begin{itemize}
    \item $x + 5 = 12$
    \item $y - 3 = 9$
    \item $2x = 10$
\end{itemize}

\textbf{Writing Algebraic Expressions:}
\begin{itemize}
    \item Write an expression for "5 more than a number $n$."
    \item Write an equation for "A number $y$ divided by 2 equals 6."
\end{itemize}

\textbf{Word Problems:}
\begin{itemize}
    \item John has 10 marbles. He buys 3 more marbles every day. How many marbles will he have after 5 days?
    \item Maria has twice as many apples as her friend. If her friend has $x$ apples, how many apples does Maria have?
\end{itemize}

\section{Real-Life Applications of Algebra}
\begin{itemize}
    \item \textbf{Budgeting:} If you have a fixed income every month and certain expenses, you can use algebra to figure out how much money you'll have left.
    \item \textbf{Travel:} If you’re driving at a certain speed, you can use algebra to calculate how long it will take you to reach your destination.
    \item \textbf{Shopping:} If you know the price of one item, you can use algebra to calculate the total cost for several of the same item.
\end{itemize}

\section{Chapter Summary}
\begin{itemize}
    \item A variable is a letter that represents an unknown number.
    \item An equation is a statement that two things are equal.
    \item You can solve equations by isolating the variable.
    \item Algebra is about recognizing patterns and relationships between numbers.
    \item The distributive property helps you simplify expressions with parentheses.
    \item You can combine like terms to make equations simpler.
\end{itemize}

Algebra allows you to explore the relationships between numbers in a more abstract way, and it will continue to be an essential tool as you progress to more advanced math topics like geometry and functions.

\section{Challenge Question:}
\begin{enumerate}[label=(\alph*)]
    \item You’re at a store, and the total cost of your items is $x$. You give the cashier \$20, and you get \$4 in change. Write an equation to represent this situation, and solve for $x$ (the total cost of your items).

\end{enumerate}