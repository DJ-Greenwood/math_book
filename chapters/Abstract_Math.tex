
\chapter{Abstract Math – Logic, Sets, and Proofs}

\section{Introduction: The Structure Behind Mathematics}
At the heart of mathematics lies a set of fundamental ideas: logic, sets, and proofs. These concepts form the foundation for nearly everything we’ve learned so far. In this chapter, we’ll take a closer look at the building blocks of mathematical reasoning and the tools mathematicians use to demonstrate whether something is true or false.

Logic, sets, and proofs may seem abstract at first, but they are powerful tools for solving problems and understanding the structure of mathematics itself. Whether you're proving a theorem, analyzing data, or coding a computer algorithm, these ideas are key to thinking clearly and solving problems systematically.

\section{What Is Logic?}
Logic is the study of reasoning. In math, logic helps us determine whether statements are true or false and how to combine these statements to make new conclusions.

\subsection{Logical Statements}
A statement in logic is a sentence that is either true or false. For example:
\begin{itemize}
    \item "2 + 2 = 4" is a true statement.
    \item "5 is greater than 10" is a false statement.
\end{itemize}
We often use symbols to represent logical statements:
\begin{itemize}
    \item \( p \) might represent "2 + 2 = 4."
    \item \( q \) might represent "5 is greater than 10."
\end{itemize}

\subsection{Logical Operations}
We can combine logical statements using logical operations, such as:
\begin{enumerate}
    \item \textbf{AND ( \(\land\) )}: Both statements must be true for the combined statement to be true.
    \begin{itemize}
        \item For example, \( p \land q \) means "p AND q."
        \item If \( p \) is true and \( q \) is false, \( p \land q \) is false.
    \end{itemize}
    \item \textbf{OR ( \(\lor\) )}: If at least one of the statements is true, the combined statement is true.
    \begin{itemize}
        \item \( p \lor q \) means "p OR q."
        \item If \( p \) is true and \( q \) is false, \( p \lor q \) is still true.
    \end{itemize}
    \item \textbf{NOT ( \(\neg\) )}: This operation negates a statement. If a statement is true, its negation is false, and vice versa.
    \begin{itemize}
        \item \( \neg p \) means "NOT p."
        \item If \( p \) is true, \( \neg p \) is false.
    \end{itemize}
\end{enumerate}

\subsection{Example}
\begin{itemize}
    \item Let \( p \) represent "It is raining."
    \item Let \( q \) represent "I have an umbrella."
\end{itemize}
If it is raining (\( p \) is true) and I have an umbrella (\( q \) is true), the statement \( p \land q \) (It is raining AND I have an umbrella) is true.

If it is not raining (\( p \) is false), the statement \( p \land q \) is false, even if I still have an umbrella.

\section{What Is a Set?}
A set is a collection of distinct objects or elements. These objects can be anything: numbers, letters, or even other sets.

\subsection{Notation}
We use curly brackets to represent a set. For example:
\begin{itemize}
    \item \( A = \{1, 2, 3, 4\} \) represents a set of four numbers.
    \item \( B = \{a, b, c\} \) represents a set of three letters.
\end{itemize}

\subsection{Elements of a Set}
The objects within a set are called elements. We use the symbol \( \in \) to indicate that something is an element of a set. For example:
\begin{itemize}
    \item \( 3 \in A \) means "3 is an element of set \( A \)."
    \item \( d \notin B \) means "d is not an element of set \( B \)."
\end{itemize}

\subsection{Types of Sets}
\begin{enumerate}
    \item \textbf{Finite Set}: A set with a limited number of elements.
    \begin{itemize}
        \item \( A = \{1, 2, 3\} \) is a finite set.
    \end{itemize}
    \item \textbf{Infinite Set}: A set with an unlimited number of elements.
    \begin{itemize}
        \item The set of all natural numbers \( \{1, 2, 3, 4, \dots\} \) is infinite.
    \end{itemize}
    \item \textbf{Empty Set}: A set with no elements, denoted by \( \emptyset \) or \( \{\} \).
\end{enumerate}

\section{Set Operations}
Just as we can perform operations on numbers, we can perform operations on sets. The most common set operations are union, intersection, and difference.

\subsection{Union ( \(\cup\) )}
The union of two sets combines all the elements from both sets. If an element is in either set (or both), it’s in the union.

Example: If \( A = \{1, 2, 3\} \) and \( B = \{3, 4, 5\} \), the union of \( A \) and \( B \) is:
\[ A \cup B = \{1, 2, 3, 4, 5\} \]

\subsection{Intersection ( \(\cap\) )}
The intersection of two sets contains only the elements that are in both sets.

Example: If \( A = \{1, 2, 3\} \) and \( B = \{3, 4, 5\} \), the intersection of \( A \) and \( B \) is:
\[ A \cap B = \{3\} \]

\subsection{Difference ( - )}
The difference between two sets contains the elements in one set but not the other.

Example: If \( A = \{1, 2, 3\} \) and \( B = \{3, 4, 5\} \), the difference of \( A \) and \( B \) (elements in \( A \) but not in \( B \)) is:
\[ A - B = \{1, 2\} \]

\section{What Is a Proof?}
A proof is a logical argument that demonstrates the truth of a mathematical statement. Proofs are the backbone of mathematics, ensuring that our conclusions are reliable and based on solid reasoning.

\subsection{Types of Proofs}
\begin{enumerate}
    \item \textbf{Direct Proof}: You start with known facts and use logical steps to arrive at the conclusion.
    \begin{itemize}
        \item Example: Prove that if \( n \) is an even number, then \( n^2 \) is even.
        \begin{itemize}
            \item Proof: If \( n \) is even, we can write \( n = 2k \) for some integer \( k \).
            \item \( n^2 = (2k)^2 = 4k^2 = 2(2k^2) \), which is clearly divisible by 2, so \( n^2 \) is even.
        \end{itemize}
    \end{itemize}
    \item \textbf{Proof by Contradiction}: You assume the opposite of what you're trying to prove, show that this leads to a contradiction, and conclude that the original statement must be true.
    \begin{itemize}
        \item Example: Prove that there is no largest prime number.
        \begin{itemize}
            \item Proof: Assume there is a largest prime number, \( p \). Now consider the number \( N = p_1 \times p_2 \times \dots \times p_n + 1 \), where \( p_1, p_2, \dots, p_n \) are all prime numbers up to \( p \).
            \item \( N \) is not divisible by any of these primes, which contradicts the assumption that \( p \) is the largest prime. Therefore, there is no largest prime number.
        \end{itemize}
    \end{itemize}
    \item \textbf{Proof by Induction}: This method is used to prove statements that hold for an infinite number of cases (often for natural numbers). It involves two steps:
    \begin{itemize}
        \item \textbf{Base Case}: Prove that the statement is true for the first value (often \( n = 1 \)).
        \item \textbf{Inductive Step}: Assume the statement is true for some \( n = k \) and then prove it is true for \( n = k + 1 \).
        \item Example: Prove that the sum of the first \( n \) natural numbers is \( \frac{n(n+1)}{2} \).
        \begin{itemize}
            \item Base Case: For \( n = 1 \), the sum is 1, which matches \( \frac{1(1+1)}{2} = 1 \).
            \item Inductive Step: Assume the formula holds for \( n = k \), meaning: \( 1 + 2 + \dots + k = \frac{k(k+1)}{2} \). Now prove it for \( n = k + 1 \):
            \[ 1 + 2 + \dots + k + (k+1) = \frac{k(k+1)}{2} + (k+1) = \frac{(k+1)(k+2)}{2} \]
            Therefore, the statement holds for \( n = k + 1 \), completing the proof by induction.
        \end{itemize}
    \end{itemize}
\end{enumerate}

\section{Mathematical Logic and Proofs in Real Life}
Logic and proofs might seem theoretical, but they have many real-world applications:
\begin{enumerate}
    \item \textbf{Computer Science}: Logic is the foundation of programming. Algorithms, conditional statements, and decision-making all rely on logical operations and reasoning.
    \item \textbf{Cryptography}: The security of modern cryptographic systems (used to protect online data) is based on proofs of mathematical statements about prime numbers and number theory.
    \item \textbf{Legal and Scientific Arguments}: Logic helps construct valid arguments, whether you're writing a legal brief or conducting a scientific experiment. Proofs and logical reasoning ensure that conclusions are reliable and based on evidence.
\end{enumerate}

\section{Practice Makes Perfect: Let’s Try Some Exercises!}
\subsection{Logic}
\begin{enumerate}
    \item Let \( p \) represent "It is sunny" and \( q \) represent "I will go for a walk." Write the following in logical notation:
    \begin{itemize}
        \item "It is not sunny, and I will go for a walk."
        \item "If it is sunny, then I will go for a walk."
    \end{itemize}
\end{enumerate}

\subsection{Sets}
\begin{enumerate}
    \item If \( A = \{1, 2, 3\} \) and \( B = \{3, 4, 5\} \), find \( A \cup B \), \( A \cap B \), and \( A - B \).
    \item If \( C = \{a, b, c\} \) and \( D = \{c, d, e\} \), find \( C \cup D \) and \( C \cap D \).
\end{enumerate}

\subsection{Proofs}
\begin{enumerate}
    \item Prove that the sum of two even numbers is always even.
    \item Use proof by contradiction to show that \( \sqrt{2} \) is irrational.
\end{enumerate}

\section{Chapter Summary}
\begin{itemize}
    \item Logic is the study of reasoning, and it helps us determine whether statements are true or false.
    \item Sets are collections of objects, and we can perform operations like union, intersection, and difference on sets.
    \item Proofs are logical arguments that demonstrate the truth of mathematical statements. They come in many forms, such as direct proofs, proof by contradiction, and proof by induction.
    \item Logic, sets, and proofs are fundamental to mathematics and have many practical applications, from computer science to cryptography.
\end{itemize}

\section*{Challenge Question}
Prove that the product of any two odd numbers is always odd. Use a direct proof to show your result.